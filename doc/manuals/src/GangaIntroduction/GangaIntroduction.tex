\documentclass{howto}

\RequirePackage{xspace}

\def\ganga {\textsc{Ganga}\xspace}
\def\python {\textsc{Python}\xspace}
\def\ipython {\textsc{IPython}\xspace}
\def\lhcb {LHC{\em b\/}\xspace}
\def\Atlas {Atlas\xspace}
\def\davinci {\textsc{DaVinci}\xspace}
\def\majorv {5}
\def\minorv {0\xspace}
\def\totalv {\majorv.\minorv}
\def\release{\ganga-\totalv}

\title{Working with \ganga \totalv}

% At minimum, give your name and an email address.  You can include a
% snail-mail address if you like.
\author{Ulrik Egede, Jakub T. Mo\'scicki}
\authoraddress{U.Egede@imperial.ac.uk, Jakub.Moscicki@cern.ch}

\begin{document}
\maketitle
\tableofcontents

\section{Migration from previous versions of \ganga}
\label{sec:migration}
This sections explains how to migrate from using \ganga 4 to \ganga 5. 
If you never used Ganga before then go to Introduction in section \ref{Introduction}.

\ganga 4 and \ganga 5 may be used concurrently and the jobs defined in
one will not be seen in the other. So you may try out \ganga 5 without
a worry that you loose your previosuly defined jobs. A downside is
that you cannot automatically migrate your jobs. You may however use
the export/load facility to restore some of your old jobs in Ganga 5
(see section \ref{sec:ExportImport}). Beware that this may sometimes 
require some manual editing if schema changed for certain objects.


\begin{notice}
The configuration (\texttt{.gangarc}) files are NOT compatible. You will get errors
using old configuration file against new ganga release.

You should create a new configuration file by running \texttt{ganga -g} 
once to avoid any potential problems. 

\end{notice}


The list of major new features is available in the Release Notes in section \ref{ReleaseNotes}.

\paragraph{The configuration changes}

\begin{enumerate}
\item \texttt{[DefaultJobRepository]} section is removed. Instead there are sections for each type of the repository: \texttt{LocalAMGA_Repository}, \texttt{RemoteAMGA_Repository},\texttt{LocalXML_Repository}.
\item  \texttt{[FileWorkspace]} is removed.
\item The \texttt{[Configuration]repositorytype} option specifies the type used in a \ganga session. 
\item The \texttt{[Configuration]gangadir} specifies the location of file workspace and the local repositories.
\item The \texttt{[Configuration]user} specifies the user name (if left blank then this will be current user name).
\item Default object attribute values are now defined in \texttt{[defaults_XXX]} sections (previously there were defined in \texttt{[XXX_Properties]}.
\end{enumerate}

At present there is not automatic conversion tool for configuration options, so you should apply the changes manually.

\paragraph{The location of the local workspace and repositories}
\begin{enumerate}
\item The local workspace is located in \begin{verbatim}$gangadir/workspace/$user/$repositorytype\end{verbatim}
\item The local repository is located in \begin{verbatim}$gangadir/repository/$user/$repositorytype\end{verbatim}
\end{enumerate}


\paragraph{Backwards-incompatible GPI changes.}

\subparagraph{Job exceptions.}

If an operation on a job fails (e.g. \texttt{j.submit()}) then there is always an exception raised. This means that the loop 
\begin{verbatim}
for j in jobs:
 j.submit()
\end{verbatim}
will not continue past the first submission failure. Use job slices instead: jobs.submit(). See section \ref{sec:collective}.

\subparagraph{The meaning of operator \texttt{jobs[]} and \texttt{jobs()}}

The operator \texttt{jobs(id)} gives back the job with the specified
id whereas \texttt{jobs[i]} gives back the i-th job in the
sequence. For exmaple \texttt{jobs[-1]} gives the last job.


\section{Introduction}
\label{Introduction}
\noindent

\ganga is a front end for job definition and management of analysis jobs to
run locally and in an distributed environment. It helps in the creation and
configuration of analysis job, submission of the jobs and finally monitors job
status and take care of saving any output. \ganga particularly aims to help
with setting up jobs that run the main ATLAS and \lhcb applications, all of
which have been developed in the Gaudi C++ framework, and are configured by
options files.

For a list of latest features  have look into Release Notes in section
\ref{ReleaseNotes}.

The \python scripting interface for working with jobs in \ganga is called the
GPI~(Ganga Public Interface). To start \ganga you simply type
\begin{verbatim}
ganga
\end{verbatim}
at the command line after entering the directory of \ganga into your PATH.

At CERN you will find \ganga installed in the directory
\begin{verbatim}
/afs/cern.ch/sw/ganga/install/5.0.x/bin
\end{verbatim}

For detailed setup and installation procedure for \Atlas and \lhcb see \url{http://cern.ch/ganga/user}.

\begin{seealso}
  If you are not familiar with \python the following URL will be a big help:
  \seeurl{http://docs.python.org/tut/tut.html}{}
\end{seealso}

Starting \ganga you should see the output below which includes a few questions
asked only the first time you start \ganga. Notice that \ganga at startup
ask you to validate your Grid proxy so you can submit jobs to the Grid. If you
do not have a certificate \ganga will continue anyway.
\begin{verbatim}
~% ganga

*** Welcome to Ganga ***
Version: Ganga-5-0-0
Documentation and support: http://cern.ch/ganga
Type help() or help('index') for online help.

This is free software (GPL), and you are welcome to redistribute it
under certain conditions; type license() for details.

It seems that you run Ganga for the first time and you do not have the config file: ~/.gangarc
Would you like to create config file with standard settings ([y]/n) ?y
Created standard config file /afs/cern.ch/user/m/moscicki/.gangarc
Press <Enter> to continue.
Ganga.GPIDev.Lib.JobRegistry       : INFO     Found 0 jobs in "jobs", completed in 0 seconds
Ganga.GPIDev.Lib.JobRegistry       : INFO     Found 0 jobs in "templates", completed in 0 seconds
Ganga.GPIDev.Credentials           : WARNING  Grid proxy has expired!
Ganga.GPIDev.Credentials           : WARNING  Renew by typing 'gridProxy.renew()' at the prompt.

\end{verbatim}
You are now ready to try out the simplest features of \ganga.


\subsection{Online help}
\label{sec:OnlineHelp}
To get online help in \ganga inside a session simply type \texttt{help()} at
the prompt. this will give you a screen as:
\begin{verbatim}
In [1]:help()

************************************

*** Welcome to Ganga ***
Version: Ganga-5-0-0
Documentation and support: http:/cern.ch/ganga
Type help() or help('index') for online help.


This is an interactive help based on standard pydoc help.

Type 'index'  to see GPI help index.
Type 'python' to see standard python help screen.
Type 'interactive' to get online interactive help from an expert.
Type 'quit'   to return to Ganga.
************************************
\end{verbatim}
Here the \texttt{index} section is where you will find help on the use of
\ganga. Below is the full list:
\begin{verbatim}
help> index
Ganga Public Interface (GPI) Index

Classes:
    File                   Represent the files, both local and remote and
                           provide an interface to transparently get access to
                           them.
    GenericSplitter        Split job by changing arbitrary job attribute.
    GaussSplitter          Create a set of Gauss jobs based on the total number
                           of jobs and the number of events per subjob
    OptionsFileSplitter    Split a jobs based on a list of option file fragments
    SplitByFiles           Splits a job into sub-jobs by partitioning the input
                           data
    ArgSplitter            Split job by changing the args attribute of the
                           application.
    DiracSplitter          Query the LFC, via Dirac, to find optimal data file
                           grouping.
    EmptyDataset           Documentation missing.
    LHCbDataset            Documentation missing.
    JobTemplate            A placeholder for Job configuration parameters.
    Job                    Job is an interface for submision, killing and
                           querying the jobs :-).
    MigrationLHCbDataset10
    LSF                    LSF backend - submit jobs to Load Sharing Facility.
    Dirac                  The backend that submits LHCb jobs to the Grid via
                           DIRAC
    PBS                    PBS backend - submit jobs to Portable Batch System.
    Condor                 Condor backend - submit jobs to a Condor pool.
    SGE                    SGE backend - submit jobs to Sun Grid Engine.
    LCG                    LCG backend - submit jobs to the EGEE/LCG Grid using
                           gLite/EDG middleware.
    Local                  Run jobs in the background on local host.
    Interactive            Run jobs interactively on local host.
    JobInfo
    Executable             Executable application -- running arbitrary programs.
    Brunel                 The Brunel Application handler
    Moore                  The Moore Application handler
    DaVinci                The DaVinci Application handler
    Gauss                  The Gauss Application handler
    Boole                  The Boole Application handler
    Gaudi                  The Gaudi Application handler
    Vetra                  The Vetra Application handler
    Root                   Root application -- running ROOT
    Euler                  The Euler Application handler
    LCGRequirements        Helper class to group LCG requirements.
    CondorRequirements     Helper class to group Condor requirements.
    LHCbDataFile           Documentation missing.
    MultipleMerger         Merger class when merges of different file types are
                           needed.
    SmartMerger            Allows the different types of merge to be run
                           according to file extension in an automatic way.
    RootMerger             Merger class for ROOT files
    TextMerger             Merger class for text


Exceptions:
     GangaAttributeError
     ProtectedAttributeError Attribute is read-only and may not be modified by
                             the user (for example job.id)
     ReadOnlyObjectError     Object cannot be modified (for example job in a
                             submitted state)
     JobError
     TreeError
     ConfigError             ConfigError indicates that an option does not exist
                             or it cannot be set.


Functions:
     license             Print the full license (GPL)
     load                Function to load previously exported Ganga objects
     export              Function to export Ganga objects to a file
     applications        return a list of all available applications, OBSOLETE:
                         use plugins("applications")
     backends            return a list of all available backends, OBSOLETE: use
                         plugins("backends")
     list_plugins        List all plugins in a given category, OBSOLETE: use
                         plugins(category)
     typename            Return a name of Ganga object as a string, example:
                         typename(j.application) -> "DaVinci"
     categoryname        Return a category of Ganga object as a string, example:
                         categoryname(j.application) -> "applications"
     plugins             List loaded plugins.
     force_job_completed manually make the job 'completed' if stuck in
                         completing state
     force_job_failed    manually make the job 'failed' if stuck in submitting
                         or completing state
     full_print          Print the full contents of a GPI object without
                         abbreviation.
     reactivate          activates the internal services previously disabled due
                         to expired credentials
     runMonitoring


Objects:
    gridProxy Grid proxy management object.
    afsToken  AFS token management object.
    jobs      default job registry
    templates default registry of job templates
    jobtree   Logical tree view of the jobs
    config    access to Ganga configuration

\end{verbatim}
At this point you can just type the name of the command to see specific help.
You exit the help system by typing Ctrl-D. To get help on an individual
command from the prompt type help and the name of the command:
\begin{verbatim}
Help on class Localhost in module Ganga.GPIDev.Base.Proxy:

class Localhost(GPIProxyObject)
 |  Local backend -- run jobs as processes on local host.
 |  
 |  Properties:
 |  
 |   actualCE Hostname where the job was submitted.. (simple property,
 |            default=None,protected,comparable)
 |  
 |   workdir  Working directory.. (simple property,
 |            default=None,protected,comparable)
 |  
 |   id       Process id.. (simple property, default=None,protected,comparable)
 |  
 |   exitcode Process exit code.. (simple property,
 |            default=None,protected,comparable)
:
:
\end{verbatim}

\section{Hello World with \ganga}
You can create jobs that runs arbitrary executables and scripts. In the first
example below we use \texttt{/bin/echo} executable which is present in most
most Unix systems to create a Hello World job.
\begin{verbatim}
In [4]:j = Job(application=Executable(exe='/bin/echo',args=['Hello World']))
In [5]:j.submit()
Ganga.GPIDev.Lib.Job               : INFO     submitting job 1
Ganga.GPIDev.Adapters              : INFO     submitting subjob 1
Ganga.Lib.Localhost                : INFO     job 1 submitted
Out[5]: 1
\end{verbatim}
The job is submitted to the default batch backend which is the local host. The
monitoring will report output each time you type return. After it has finished
you can get the output.
\begin{verbatim}
In [6]:
Ganga.Lib.Localhost                : INFO     Local job 1 status changed to running, pid=9642
Ganga.Lib.Localhost                : INFO     Local job 1 finished with exitcode 0

In [6]:outfile = file(j.outputdir+'stdout')
In [7]:print outfile.read()
Hello World
In [8]:j.peek('stdout')
\end{verbatim}

The \texttt{peek()} executes an external command to show the files in
the job output directory. The command may be customized in
\texttt{[File_Associations]} configuration section.

You may also invoke the shell commands directly from \ganga.

\begin{verbatim}
In [9]:!cat $j.outputdir/stdout
Hello World
\end{verbatim}
The exclamation mark before the cat command above is the way to tell \ipython
that you issue a shell command. The dollar sign indicates that you return to a
python command.  Your first \ganga job was successfully completed!  If you
have problems with the job you should check the other files in the directory
given by \texttt{j.outputdir}.

\begin{seealso}
  \ipython is a third party product that makes it much nicer to work with the
  \python prompt. It provies shortcuts, command line history, colouring etc.
  See \seeurl{http://ipython.scipy.org}{} for further information.
  documentation.
\end{seealso}



\section{Running executables with \ganga}

The  HelloWorld example  uses  the {\tt  /bin/echo}  command which  is
available  on  any computer.  However  if you  want  to  run your  own
executable (a binary or a script) you should tell \ganga that you want
to use  this executable  {\it file} not  a simple command.  Your job
should be specified like this:

\begin{verbatim}
j = Job()
j.application = Executable(exe=File('~/somescript'), args=['1','2','3'])
j.submit()
\end{verbatim}

Ganga  will  make  sure  that  the file  {\tt$\tilde{}$/somescript}  will  be
correctly copied  and executed  with the command  {\tt ./somescript  1 2
3}. This mechanism is called an {\it input sandbox} and is designed to
handle small files (up to few MB). More information about handling files
in section \ref{sec:files}.

\begin{notice}
Do not confuse the {\tt File} class
which  is  a \ganga-specific  class  to  ship  files remotely  with  a
python's  {\tt file}  built-in function  which is  used  to read/write
local files.
\end{notice}


\section{More on \ganga job objects}
Each job in \ganga has a unique number (identifier).  \ganga jobs are
automatically stored on in a {\em job registry}. You can use the job id number
to find jobs submitted in previous \ganga sessions.
\begin{verbatim}
In [10]: print j.id
1
\end{verbatim}

Print the summary of a job by just typing the object on a line of its own:
\begin{verbatim}
In [10]:j
Out[10]: Job (
 status = 'completed' ,
 name = '' ,
 inputdir = '/afs/cern.ch/user/u/uegede/gangadir/workspace/Local/1/input/' ,
 outputdir = '/afs/cern.ch/user/u/uegede/gangadir/workspace/Local/1/output/' ,
 id = 1 ,
 outputsandbox = [] ,
 splitter = None ,
 inputsandbox = [ ] ,
 application = Executable (
    exe = '/bin/echo' ,
    env = {} ,
    args = ['Hello World'] 
    ) ,
 subjobs = [ ] ,
 inputdata = None ,
 backend = Local (
    actualCE = 'lxplus059.cern.ch' ,
    workdir = '/tmp/@8263.2' ,
    id = 9642 ,
    exitcode = 0 
    ) 
 ) 
\end{verbatim}

\begin{notice}
Sometimes the full printout of job attributes may get quite lengthy, especially
if a job has many files or subjobs. So by default \ganga does a summary print in
a way similar to the standard python \texttt{repr} module. To get the
full print of a job (or any other \ganga object) you may use the \texttt{fullprint()} function:
\begin{verbatim}
full_print(j)
\end{verbatim}
\end{notice}

You can access the properties of the job directly as attributes and use the
programmatic properties of \python. For example:
\begin{verbatim}
In [14]: j.application
Executable {
 exe = '/bin/date'
 }
In [17]: if j.status == 'completed':
   ....:     print 'Great, I completed job',j.id,'.'
   ....:
Great, I completed job 1 .
\end{verbatim}

All jobs are stored in the job registry, called \texttt{jobs}:
\begin{verbatim}
In [30]:jobs
Out[30]: Statistics: 4  jobs
--------------
#   id      status        name    application   backend   backend.actualCE  
#    1   completed                 Executable     Local  lxplus059.cern.ch  
#    2   completed      Hello2     Executable     Local  lxplus059.cern.ch  
#    3         new  myAnalysis        DaVinci     Dirac                     
#    4     running      Hello3     Executable     Local  lxplus059.cern.ch  
\end{verbatim}

You may retrieve job objects from a registry using their ids:
\begin{verbatim}
In [27]: print jobs(18).status
completed
In [28]: for job in jobs:
   ....:     print job.id
   ....:
18
19
20
21
\end{verbatim}

If a job is in the \texttt{submitted} or \texttt{running} state you can kill
it. Otherwise you will get an error.
above.
\begin{verbatim}
In [32]:jobs(4).kill()
Ganga.GPIDev.Lib.Job               : INFO     killing job 4
Out[32]: 1

In [33]:jobs(1).kill()
Ganga.GPIDev.Lib.Job               : INFO     killing job 1
Ganga.GPIDev.Lib.Job               : WARNING  cannot kill job which is in 'completed' state
\end{verbatim} 
Notice that from the \ipython prompt tab expansion works to explore available
methods on an object so you can do:
\begin{verbatim}
In [34]:j.
j.application    j.fqid           j.inputdir       j.merger         j.outputsandbox  j.splitter
j.backend        j.id             j.inputsandbox   j.name           j.peek           j.status
j.copy           j.info           j.kill           j.outputdata     j.remove         j.subjobs
j.fail           j.inputdata      j.master         j.outputdir      j.resubmit       j.submit
\end{verbatim} 
to see the available commands. Attributes starting by an underscore are internal and are not shown.
As an example you can see there is a remove method which will remove the job from the registry
(after killing it first if running). 

Notice that jobs in \ganga are persisted. If you quit \ganga (type Ctrl-D at
the prompt) and reenter you will find all your jobs again.

\section{Complex Jobs}
\ganga is implemented as a small core that provides you with the syntax as
well as the functionality of storing and retrieving jobs. Everything else is
achieved through particular plugins. Many of these are experiment specific
(like Athena jobs for ATLAS and \davinci jobs for \lhcb) so only a few general
ones will be described here. Look in the specific manuals for experiment
specific plugins.

\subsection{Submission backends}
Job submission backend defines where your job will be run. You can run the
same job you ran locally just by attaching a different backend. The
\texttt{Local}, \texttt{LSF} and \texttt{LCG} handlers are available in all
\ganga sessions. As an example we can start a set of jobs to run on LSF.
\begin{verbatim}
In [58]:for i in range(10):
   ....:     j = Job(application=myexe, backend='LSF', name='LSFHostname')
   ....: 
In [59]:jobs[-10:].submit() #submit last 10 created jobs

< wait for jobs to finish >

In [85]:for j in jobs[-10:]:
   ....:     !grep '^lxb' $j.outputdir/stdout
   ....:     
lxb1456.cern.ch
lxb0040.cern.ch
lxb0422.cern.ch
lxb0422.cern.ch
lxb1002.cern.ch
lxb1270.cern.ch
lxb6619.cern.ch
lxb5407.cern.ch
lxb0581.cern.ch
lxb6084.cern.ch
\end{verbatim}

To get a list of available backends in your session type \texttt{plugins("backends")}.
The output will depend on the experiment specific plugins you have loaded.
\begin{verbatim}
In [86]:plugins("backends")
Out[86]: ['LSF', 'LCG', 'Dirac', 'gLite', 'PBS', 'Condor', 'Local']
\end{verbatim}
While LSF might be available in \ganga as a backend it of course still depends
on that you have a local LSF batch system. There is no magic here.

Each backend object will store an \texttt{id} for the job when it is running.
For the local handler it is the process id, for LSF it is the LSF job id etc.
You can use this to get access to further information. As an example you can
get the output from a running LSf job with the \texttt{bpeek} command like:
\begin{verbatim}
In [67]:!/usr/local/lsf/bin/bpeek $j.backend.id
<< output from stdout >>
@(#)CERN job starter $Date: 2008-07-17 16:40:41 $
Working directory is </pool/lsf/uegede/767136> on <lxb6084.cern.ch>

lxb6084.cern.ch
DEBUG: cwd= /pool/lsf/uegede/767136
DEBUG: ls = ['1142590441.767136.out', '1142590441.767136.err', '.lsf.start', '__jobscript__']
DEBUG: Job finshed with exit code:  0

Job finished at Fri Mar 17 11:16:01 CET 2006 on node 
 under linux version Scientific Linux CERN Release 3.0.6 (SL)


CERN statistics: This job used 0:00:00 NCU hours (0 NCU seconds)

CERN statistics: This job used 0:00:00 KSI2K hours (0 KSI2K seconds)

KSI2K = kilo-SpecInt2000 benchmark units = 1.00 NCU

<< output from stderr >>
\end{verbatim}

\subsection{Applications}
In principle it would be possible to perform all analysis simply by writing
scripts that are executed via the simple \texttt{Executable} application
backend as in the examples above. For a complicated analysis job this is far
from trivial though. To help on this \ganga provides a set of specific
application backends. For \lhcb there are applications for all the standard
\textsc{Gaudi} jobs (\textsc{Gauss}, \textsc{Bool}, \textsc{Brunel},
\textsc{DaVinci}) while for ATLAS there is an \textsc{Athena} application.

Too see the applications that are available in your current \ganga session
simply type \texttt{plugins('applications')}. Notice that the output will depend on the
experiment specific handlers you have loaded.
\begin{verbatim}
In [76]: plugins('applications')
Out[76]: ['Boole', 'Gaudi', 'Executable', 'Brunel', 'Euler', 'DaVinci', 'Gauss', 'Moore']
\end{verbatim}

The application interface provides information necessary to configure a job in
a way which is the most convenient for a given application. For example, it may
automatically find the executable file using some conventions, it may define
application-specific options etc.  Each job object has a corresponding
application interface.

Typically applications live inside the job objects but they may also be
created separately. Below is given an \lhcb example:
\begin{verbatim}
In [77]:dv = DaVinci(version='v12r16')
In [78]:print dv
DaVinci (
 version = 'v12r16' ,
 extraopts = None ,
 configured = 0 ,
 package = 'Phys' ,
 cmt_user_path = '/afs/cern.ch/user/u/uegede/cmtuser' ,
 masterpackage = None ,
 optsfile = File (
    name = '' ,
    subdir = '.' 
    ) 
 ) 
\end{verbatim}

Application objects may be assigned to jobs.  In that case all properties of
the application will be copied into the job.application object. In \ganga the
copying of complex structures such as applications within the job context is
always done by value. This guarantees the consistency of the jobs:
\begin{verbatim}
In [79]:j = Job(application=dv)
In [80]:j.application.version='v16r0'
In [81]:print dv.version
v12r16
In [82]:print j.application.version
v16r0
\end{verbatim}

Within the job context you can also use string names of the applications to
create new objects. The example below will create an empty
\code{Executable()} application object and assign it to the job:
\begin{verbatim}
>>> j.application = 'Executable'
\end{verbatim}

\begin{notice}
  Not all applications can be matched with all backends. This is to not
  enforce the system to artificially support situations that wouldn't make
  sense anyway like submitting an \lhcb \davinci job to the ATLAS production
  system. If you combine an application with a non-compatible backend you will
  get an error at submission time.
\end{notice}

\subsection{Input data and input sandbox}
You may specify additional files which will be copied to the worker node as
the inputsandbox. This mechanism works for small files (less than a few MB).
Submission backends may impose limitations on the size of the sandbox, so
occasionally you may get an error if you exceed the quota. Also remember that
\ganga keeps the input sandbox for all jobs in the file workspace (typically in
\texttt{$\tilde{}$/gangadir/workspace}) so you may run into quota problems.

The input data is an application-specific description of the large data files or
datasets which are required for the job. The transfer of input data is handled
in a special way and typically there are no restrictions on the size. The
input data may also influence the brokering process on the Grid your job may go
the sites where the data is found.

You are able to find further details about how the input data and input
sandbox parameters are used in the documentation of individual applications
and backends.

\subsection{Handling files and input/output sandbox}
\label{sec:files}

The Executable application allows to specify sandbox files directly as executables or the arguments:

\begin{verbatim}
j = Job()
j.application = Executable(exe=File('/opt/anotherscript'), args=['-d',File('/etc/x')])
j.submit()
\end{verbatim}

In this example \ganga has two files to copy around in the sandbox and
the  executed command  is {\tt  anotherscript  -d x}.  Both files  are
copied to the same directory on the worker node.

\begin{notice}
Note the difference between \texttt{File} and \texttt{file}.


\texttt{File('name')} is \ganga-specific object which just tells
\ganga to put the file into the sandbox, whereas  \texttt{file('name')} is a
standard python file object which opens the file for reading.  The two
flavours of file objects are NOT interchangeable.
\end{notice}


What if  you just need to  copy the file  which is not an  argument of
Executable or you are dealing with a different application? You simply
add it to the inputsandbox, like in this example:

\begin{verbatim}
j = Job()
j.application = Executable(exe=File('do_something_script'), args=[])
j.inputsandbox = [File('~/extra_file')]
j.submit()
\end{verbatim}

Note, that  if you  use the {\tt  File('do_something_script')} object,
the file  name will be  automatically converted to its  absolute path.
This is useful, because the {\tt File}  object will still be usable, even if
you change the  current directory. If you are  writing scripts you may
want to disable this feature  by assigning the file name directly like
this: {\tt File(name='a/b/myfile')}.

Also note,  that if  you assign strings  to {\tt  j.inputsandbox} they
will  be automatically  converted to  {\tt File}  objects. So  the two
lines below are equivalent:

\begin{verbatim}
j.inputsandbox = [File(extra_file')]
j.inputsandbox = ['extra_file']
\end{verbatim}

Finally you  may control the file  destination on the  worker node. By
default the  files are copied  to the current working  directory ({\tt
'.'}) but  specifying a File's  {\tt subdir} attribute you  may change
it.

The {\tt  outputsandbox} specifies which  files should be  copied from
the worker  node to  the submitter  machine and is  a list  of strings
(contrary  to the {\tt  inputsandbox} which  is a  list of  {\tt File}
objects). The {\tt  outputsandbox} individual names or shell patterns, for example:

\begin{verbatim}
j.outputsandbox = ['b.dat','a*.txt']
\end{verbatim}

The {\tt b.dat} file will be copied (if exists) as well as all files which match the 
{\tt a*.txt} pattern.


\section{Job manipulation}

\subsection{Copying jobs}
You may use an existing job object to create a new job. The first non-named
object in the creation of a new object will be used as a copy
constructor. Hence the two examples below will both create a copy of the old
job, but with a different name:
\begin{verbatim}
In [1]:j = Job(name='original')
In [2]:jcopy1 = j.copy()
In [3]:jcopy1.name='Copy1'
In [4]:jcopy2 = Job(j,name='Copy2')
In [5]:print jobs
Statistics: 3  jobs
--------------
#   id      status        name   application   backend 
#   57         new    original    Executable     Local 
#   58         new       Copy1    Executable     Local 
#   59         new       Copy2    Executable     Local 
\end{verbatim}

\subsection{Failing jobs}
It may be sometimes useful to explicitly mark the job as failed e.g. to indicate wrong output or some other malfunction. 

A submitted or running job must be first killed.

\begin{verbatim}
j.fail()
\end{verbatim}

\subsection{Job Templates}
Job templates in \ganga are very similar to normal jobs except that they
cannot be submitted. Instead they are a way of predefining and saving job
parameters for future use. They are stored in a separate register called
\texttt{templates}. The example below creates a template.

\begin{verbatim}
In [6]:dv=DaVinci(version='v12r16')
In [7]:dv.masterpackage='Tutorial/UKAnalysis/v1r0'
In [8]:dv.cmt_user_path='/afs/cern.ch/user/u/uegede/cmttest'
In [9]:dv.optsfile='myopts.opts'
In [10]:t = JobTemplate(application=dv,name='Tutorial')
In [11]:print templates
Statistics: 1  templates
--------------
#   id      status        name   application  backend
#    1    template    Tutorial       DaVinci    Local
\end{verbatim}

You can then in the same or in a future \ganga session create jobs based on
the template.
\begin{verbatim}
In [12]:tdv=templates[1]
In [13]:j1 = Job(tdv,name='JobFromTemplate')
In [14]:j2 = Job(tdv,backend=LSF(),name='JobWithModifiedBackend')
In [15]:print j1,j2
Job (
 status = 'new' ,
 name = 'JobFromTemplate' ,
 id = 60 ,
:
 application = DaVinci (
    version = 'v12r16' ,
:
    ) ,
 backend = Local (
:
    ) 
 ) 
 Job (
 status = 'new' ,
 name = 'JobWithModifiedBackend' ,
 id = 61 ,
:
 application = DaVinci (
    version = 'v12r16' ,
:
    ) ,
 backend = LSF (
:
    ) 
 ) 
\end{verbatim}

% It is also useful to refer to templates by name. The command
% \begin{verbatim}
% In [28]: list = templates['Tutorial']
% \end{verbatim}
% will give you a list of job templates with the name Tutorial. If you only have
% one (or want the first one with this name), the following is an easy way to
% submit a job based on this template:
% \begin{verbatim}
% In [29]: Job(templates['Tutorial'][0]).submit()
% \end{verbatim}

\subsection{Removing jobs}
You may see the status of all your active jobs as in the example below:
\begin{verbatim}
In [1]: print jobs
Statistics: 5  jobs
--------------
  ID     status           name
# 61        new    original
# 62        new       Copy1
# 63        new       Copy2
# 64        new  JobFromTemplate
# 65        new  JobWithModifiedBackend
\end{verbatim}

If you consider that your jobs occupy too much space you may easily remove
them and associated directories from disk:
\begin{verbatim}
In [2]: jobs(64).remove()
Ganga.GPIDev.Lib.Job               : INFO     removing job 64
\end{verbatim}
To remove all jobs in the registry you can use the \texttt{remove} command
\begin{verbatim}
In [4]: jobs.remove()
In [5]: print jobs
Statistics: 0  jobs
\end{verbatim}
Take care; this acts like \emph{rm -rf *} in your file system.

\subsection{Export and import of jobs}
\label{sec:ExportImport}
It is possible in \ganga to export jobs into a text format and then later read
them back in. This is not a replacement for your repository where job
definitions are stored between \ganga sessions, but may serve as a useful way
to exchange job definitions with other users. The following example shows the
\texttt{export} of a job, the context of the text file it is saved in and a
subsequent \texttt{load} of the job.
\begin{verbatim}
In [16]:j = Job(application=DaVinci(), backend=Dirac())
In [17]:export(j,'myjob.txt')
In [18]:!cat myjob.txt
#Ganga# File created by Ganga - Fri Mar 17 11:37:48 2006
#Ganga#
#Ganga# Object properties may be freely edited before reloading into Ganga
#Ganga#
#Ganga# Lines beginning #Ganga# are used to divide object definitions,
#Ganga# and must not be deleted

#Ganga# Job object (category: jobs)
Job (
 name = '' ,
 outputsandbox = [] ,
 splitter = None ,
 inputsandbox = [ ] ,
 application = DaVinci (
    version = 'v15r0' ,
    extraopts = None ,
    package = 'Phys' ,
    cmt_user_path = '/afs/cern.ch/user/u/uegede/cmtuser' ,
    masterpackage = None ,
    optsfile = File (
       name = '' ,
       subdir = '.' 
       ) 
    ) ,
 inputdata = None ,
 backend = Dirac (
    )  ) 
\end{verbatim}
Notice that the file is written in a clear text format. You can edit this in a
text editor and maybe use it as a starting point for writing scripts. 

You can also \texttt{export} the whole or part of your repository like
\texttt{export(jobs)}. See the help text for \texttt{export} and \texttt{load}
for all the available options (type \texttt{help load} at the \ipython
prompt).



\section{Tracking jobs}
\label{sec:tracking}

You may see the status of all your active jobs as in the example below:
\begin{verbatim}
In [14]:jobs
Out[14]:
Job slice:  jobs (6 jobs)
--------------
# fqid      status        name   subjobs      application          backend                               backend.actualCE
#    0   completed                             Executable            Local                              lxplus232.cern.ch
#    1      failed         old                 Executable              LSF
#    2   completed   important                 Executable            Local                              lxplus232.cern.ch
#    3   completed   important                 Executable            Local                              lxplus232.cern.ch
#    4         new         old                 Executable              LCG
#    6         new  a special one!             Executable              LSF
\end{verbatim}

The easiest way to access a job is using the id number:

\begin{verbatim}
In [15]: print jobs(1)
Job (
 status = 'failed' ,
 name = 'old' ,
 id = 1 ,
 ...
 )

\end{verbatim}

The job name may be used as an additional label. A name does not have
to be unique. However if the name is unique there is a shortcut which
allows to find it easily:

\begin{verbatim}

In [16]:j = jobs['a special one!']
In [17]:print j
Out[17]: Job (
 status = 'new' ,
 name = 'a special one!' ,
 id = 6 ,
 ...

In [18]:jobs['old']
JobAccessError: job "old" not unique
\end{verbatim}

To get all jobs with a given name, you may use the select method:

\begin{verbatim}
In [25]:old_jobs=jobs.select(name='old')
In [26]:print old_jobs
Job slice:  jobs.select(minid=None,maxid=None,name='old') (2 jobs)
--------------
# fqid      status        name   subjobs      application          backend                               backend.actualCE
#    1      failed         old                 Executable              LSF
#    4         new         old                 Executable              LCG
\end{verbatim}

See section \ref{sec:collective} for more on select operations.

Another way of accessing jobs is using the standard python list indexing and slicing. This is the way to access the first and the last job:

\begin{verbatim}
In [50]:first=jobs[0]
In [51]:print first
Job (
 status = 'completed' ,
 name = '' ,
 id = 0
...)

In [52]:last=jobs[-1]
In [53]:print last
Job (
 status = 'new' ,
 name = 'a special one!' ,
 id = 6
 ... )
\end{verbatim}

The two first jobs

\begin{verbatim}
In [59]:first2=jobs[:2]
In [60]:print first2
Out[60]:
Job slice:  jobs[0:2] (2 jobs)
--------------
# fqid      status        name   subjobs      application          backend                               backend.actualCE
#    0   completed                             Executable            Local                              lxplus232.cern.ch
#    1      failed         old                 Executable              LSF
\end{verbatim}

The last two jobs:

\begin{verbatim}
In [61]:last2=jobs[-2:]
In [62]:print last2
Out[62]:
Job slice:  jobs[4:2147483647] (2 jobs)
--------------
# fqid      status        name   subjobs      application          backend                               backend.actualCE
#    4         new         old                 Executable              LCG
#    6         new  a special one!             Executable              LSF
\end{verbatim}

The jobs access notation is optimized to give maximum flexibility and convenience.

\begin{notice}
Syntax to access individual jobs:
\begin{enumerate}
\item \texttt{jobs[i]}: get i-th job (counting from 0) (*)
\item \texttt{jobs['name']}: get job by name if the name is unique (**)
\item \texttt{jobs(id)}: get the job by the id number (**)
\item \texttt{jobs((id1,id2)} is equivalent to \texttt{jobs(id1).subjobs(id2)} (**)
\item \texttt{jobs('id1.id2')} like above (**)
\end{enumerate}
\begin{itemize}
\item [*] raise JobAccessIndexError if not found. This exception is derived from IndexError.
\item [**] raise JobAccessError if not found.
\end{itemize}
\end{notice}

\begin{notice}
Syntax to access job slices:
\begin{enumerate}
\item \texttt{jobs[i:k]}: get a slice like for the list
\item \texttt{jobs.select(min_id,max_id)}: get a slice for all jobs for which \texttt{min_id<=id<=max_id}
\end{enumerate}
If no jobs are found in the selected range, the empty selection is returned.

For a complete syntax of select operation see section \ref{sec:collective}.
\end{notice}
\subsection{Job tree}
\label{sec:jobtree}
The \texttt{jobtree} object in \ganga gives you the possibility to organise
jobs in a directory structure. Look at the session below to get a feeling for
what is possible.
\begin{verbatim}
In [1]:jobs
Out[1]: Statistics: 6  jobs
--------------
#   id  status        name  application    backend
#   57     new    original   Executable      Local
#   58     new       Copy1   Executable      Local
#   59     new       Copy2   Executable      Local
#   60     new  JobFromTem      DaVinci      Local
#   61     new  JobWithMod      DaVinci        LSF
#   62     new  MyAnalysis      DaVinci      Dirac
In [2]:jobtree
Out[2]: 
[Folders]: 0
[Jobs]: 0
In [3]:jobtree.mkdir('testJobs')
In [4]:jobtree.mkdir('myAnalysis')
In [5]:jobtree.cd('testJobs')
In [6]:jobtree.add(jobs(57))
In [7]:jobtree.cd('/myAnalysis')
In [8]:jobtree.add(jobs(60))
In [9]:jobtree.add(jobs(61))
In [10]:jobtree.add(jobs(62))
In [11]:jobtree.cd('/')
In [12]:jobtree.printtree
------->jobtree.printtree()
  [/]
    |-[myAnalysis]
        |-60
        |-61
        |-62
    |-[testJobs]
        |-57
In [13]:jobtree.cd('/myAnalysis')
In [14]:for j in jobtree.getjobs(): print j.name
   ....: 
JobFromTemplate
JobWithModifiedBackend
 MyAnalysis
\end{verbatim}

Jobs are stored in the jobtree by their index and you can think of it as a
softlink. This also means that jobs can be placed in several folders at the
same time. If you remove a job from the repository the references in the
jobtree will also automatically disappear. The jobtree is persisted in
between \ganga sessions.

To see the full list of commands on the jobtree type \texttt{help
  type(jobtree)} on the command line.
 




\section{Collective job operations}
\label{sec:collective}

Consider the following jobs:

\begin{verbatim}
In [14]:jobs
Out[14]:
Job slice:  jobs (6 jobs)
--------------
# fqid      status        name   subjobs      application          backend                               backend.actualCE
#    0   completed                             Executable            Local                              lxplus232.cern.ch
#    1      failed         old                 Executable              LSF
#    2   completed   important                 Executable            Local                              lxplus232.cern.ch
#    3   completed   important                 Executable            Local                              lxplus232.cern.ch
#    4         new         old                 Executable              LCG
#    6         new  a special one!             Executable              LSF
\end{verbatim}

A subset of jobs may be selected using simple criteria such as name, status, application or backend:

\begin{verbatim}
In [29]:new_jobs=jobs.select(status='new')

In [30]:print new_jobs
Job slice:  jobs.select(minid=None,maxid=None,status='new') (3 jobs)
--------------
# fqid      status        name   subjobs      application          backend                               backend.actualCE
#    3         new   important                 Executable            Local
#    4         new         old                 Executable              LCG
#    6         new  a special one!             Executable              LSF
\end{verbatim}

\begin{notice}
Full syntax of select operation:

\texttt{jobs.select(min_id,max_id,attr=value,...)} yields a selection of jobs \texttt{j} for which:
\begin{enumerate}
\item \texttt{min_id<=j.id<=max_id} (min_id or max_id may be ommited)
\item \texttt{j.attr==value} (the supported attributes are \texttt{name,status,backend,application})
\end{enumerate}

Multiple attributes may be specified, each attribute narrows the selection (AND).
\end{notice}


The jobs in the selection may be manipulated in a single operation:

\begin{verbatim}
In [31]:new_jobs.submit()
Ganga.GPIDev.Lib.Job               : INFO     submitting job 3
Ganga.GPIDev.Adapters              : INFO     submitting job 3 to Local backend
Ganga.GPIDev.Lib.Job               : INFO     job 3 status changed to "submitted"
Ganga.GPIDev.Lib.Job               : INFO     submitting job 4
Ganga.GPIDev.Adapters              : INFO     submitting job 4 to LCG backend
...
\end{verbatim}

The advantage of calling methods on the selection of jobs is not only
the short notation but also a possibility of optimizing certain
operations internally. The optimization depends however on the
implementation details of the submission backends and may not always
be possible.

Here the list of all supported operations:

\begin{enumerate}
\item \texttt{submit(keep_going=True)}
\item \texttt{copy(keep_going=True)}
\item \texttt{kill(keep_going=True)}
\item \texttt{resubmit(keep_going=True)}
\item \texttt{remove(keep_going=True,force=True)}
\item \texttt{fail(keep_going=True,force=True)}
\end{enumerate}

Here is how to mark all jobs as failed (whenever this operation is permitted):

\begin{verbatim}

In [2]:jobs.fail()
Ganga.GPIDev.Lib.Job               : INFO     job 0 status changed to "failed"
Ganga.GPIDev.Lib.Job               : INFO     job 1 status changed to "failed"
Ganga.GPIDev.Lib.Job               : INFO     job 2 status changed to "failed"
Ganga.GPIDev.Lib.Job               : INFO     job 3 status changed to "failed"
Ganga.GPIDev.Lib.Job               : WARNING  JobStatusError:  ('forbidden status transition of job 4 from "new" to "failed"',)
Ganga.GPIDev.Lib.Job               : WARNING  JobStatusError:  ('forbidden status transition of job 6 from "new" to "failed"',)

In [3]:jobs
Out[3]:
Job slice:  jobs (6 jobs)
--------------
# fqid      status        name   subjobs      application          backend                               backend.actualCE
#    0      failed                             Executable            Local                              lxplus232.cern.ch
#    1      failed         old                 Executable              LSF
#    2      failed   important                 Executable            Local                              lxplus232.cern.ch
#    3      failed   important                 Executable            Local                              lxplus232.cern.ch
#    4         new         old                 Executable              LCG
#    6         new  a special one!             Executable              LSF

\end{verbatim}

If the \texttt{keep_going} option is set to True (default) then the operation continues despite possible errors. Consider this example:

\begin{verbatim}
In [11]:jobs.submit(keep_going=True)
Ganga.GPIDev.Lib.Job               : WARNING  cannot submit job 0 which is in 'failed' state
Ganga.GPIDev.Lib.Job               : WARNING  cannot submit job 1 which is in 'failed' state
Ganga.GPIDev.Lib.Job               : WARNING  cannot submit job 2 which is in 'failed' state
Ganga.GPIDev.Lib.Job               : WARNING  cannot submit job 3 which is in 'failed' state
Ganga.GPIDev.Lib.Job               : INFO     submitting job 4
Ganga.GPIDev.Adapters              : INFO     submitting job 4 to LCG backend
Ganga.GPIDev.Lib.Job               : WARNING  JobManagerError: error during submit ... reverting job 4 to the new status
Ganga.GPIDev.Lib.Job               : INFO     job 4 status changed to "new"
Ganga.GPIDev.Lib.Job               : INFO     submitting job 6
Ganga.GPIDev.Adapters              : INFO     submitting job 6 to LSF backend
Ganga.GPIDev.Lib.Job               : INFO     job 6 status changed to "submitted"
\end{verbatim}

On the other hand, if setting the \texttt{keep_going} option to False, makes the whole operation abort at the first error:

\begin{verbatim}
In [10]:jobs.submit(keep_going=False)
Ganga.GPIDev.Lib.Job               : WARNING  cannot submit job 0 which is in 'failed' state

JobError: cannot submit job 0 which is in 'failed' state
\end{verbatim}

Additionally, the \texttt{force=True} flag may be used to always override the job status for \texttt{remove} and \texttt{fail} operations.

The collective operations and selections may also be performed on subjobs. For example the following provides convenient shorthand notation:

\begin{verbatim}
j.subjobs.resubmit()
j.subjobs.select(status='killed').fail()
\end{verbatim}
 

\section{Job splitting and merging}

Often computing tasks are far too big to be carried out on a single CPU within
an acceptable amount of time. To solve this we can divide a job into bits that
can execute in parallel on different worker nodes. To support this way of
working \ganga has the concept of job splitting. You define a \emph{master
  job} that contains the complete job and the provide a \emph{splitter} that
has the knowledge to divide the problem into many \emph{sub jobs}. The typical
application is the analysis of a very large dataset. What the splitter does in
this case is to divide the dataset into may smaller chunks that are analysed
in individual jobs.


\subsection{Subjobs}
\label{sec:subjobs}

In the core of \ganga there is a simplistic splitter called
\texttt{ArgSplitter}. This splitter simply takes an list of args
creates a sub job for each of them, assuming that the application has
the args attribute (which is the case of Executable and Root).  Below
is an example that creates a master job with 3 sub jobs. The sub jobs
simply prints the letters \texttt{A}, \texttt{B}, and \texttt{C}
\begin{verbatim}
In [21]:s = ArgSplitter(args=[['A'],['B'],['C']))
In [22]:j = Job(splitter=s)
In [23]:j.submit()
Ganga.GPIDev.Lib.Job               : INFO     submitting job 164
Ganga.GPIDev.Adapters              : INFO     submitting subjob 16400001
Ganga.Lib.Localhost                : INFO     job 16400001 submitted
Ganga.GPIDev.Adapters              : INFO     submitting subjob 16400002
Ganga.Lib.Localhost                : INFO     job 16400002 submitted
Ganga.GPIDev.Adapters              : INFO     submitting subjob 16400003
Ganga.Lib.Localhost                : INFO     job 16400003 submitted
Out[23]: 1
\end{verbatim}
Notice how the single submit command on the master job causes the 3 sub jobs
to be created and submitted. To look at the output of the sub jobs we can
access them as a list from the master job:
\begin{verbatim}
In [24]:for sj in j.subjobs:
   ....:     j.peek('stdout','cat')
   ....:     
A
B
C
\end{verbatim}

The status of a master job is determined by the status of the sub jobs. If any
jobs are in the \texttt{running} state, the master will be in the
\texttt{running} state. Only once all sub jobs have completed will the master
go into \texttt{completed} state.

For splitting according to datasets which is probably more useful, the only
implementation so far is for \lhcb Gaudi jobs. See the \lhcb specific manual
for how this is used.

To get the list of available splitters type \texttt{plugins('splitters')}:
\begin{verbatim}
In [1]:plugins('splitters')
Out[1]: ['GenericSplitter', 'ArgSplitter']
\end{verbatim}

\subsection{Merging subjobs}

Users often partition jobs, either manually, or automatically with a
splitter, so that the output files from several jobs needs to be
combined at the later stage.

\ganga provides a few predefined mergers which may be listed by typing \texttt{plugins('mergers')}:

\begin{verbatim}
In [1]:plugins('mergers')
Out[1]: ['MultipleMerger', 'SmartMerger', 'RootMerger', 'TextMerger']
\end{verbatim}

Consider the job defined in the previous section:

\begin{verbatim}
In [2]:s = ArgSplitter(args=[['A'],['B'],['C']))
In [3]:j = Job(splitter=s)
\end{verbatim}

The automatic merger of the standard output may be enabled as follows:

\begin{verbatim}
In [4]:j.merger=TextMerger()
In [5]:j.merger.files=['stdout']
In [6]:j.merger.ignorefailed = True
In [7]:j.submit()
\end{verbatim}

The merging of subjobs is done automatically when the status of the
master job changes to completed. If the ignorefailed flag has been
set, then it will also be run when the master jobs status is set to
failed.

The output of the merge will be put in the master jobs outputdir
unless the merger tool has its outputdir variable set, in which case
the specified location will be used. Any files output will be
decorated with the master job ID to avoid clashes.

\begin{verbatim}
In [17]:j.peek()
total 3.0K
-rw-r--r--  1 moscicki sf  903 Jun 11 13:08 stdout.merge_summary
-rw-r--r--  1 moscicki sf 1.2K Jun 11 13:08 stdout
\end{verbatim}

\subsection{Merging collections of jobs}

Manual merging allows a user to combine the output of arbitrary collection of jobs. 

Consider the following jobs:

\begin{verbatim}
In [3]:j = Job(name='merge-demo')
In [4]:j = Job(name='merge-demo')
In [5]:demojobs = j.select(name='merge-demo')
In [6]:demojobs.submit()
\end{verbatim}

The merger object does the rest. Note that the outputdir must be
specified, eventually having a sensible default in .gangarc.

\begin{verbatim}
In [8]:m = TextMerger()
In [10]:m.files=['stderr']
In [13]:m.merge(demojobs,outputdir = '~/merge_dir')
Out[13]: True

In [14]:!ls -l ~/merge_dir
total 2
-rw-r--r--  1 kuba z5 274 Jun 11 13:47 stderr
-rw-r--r--  1 kuba z5 203 Jun 11 13:47 stderr.merge_summary
\end{verbatim}

No name decoration will be done, so if an operation would have caused
overwriting, then an error will be raised.  If the user wants to force
overwriting, they can set the overwrite flag on the Merge object:
\texttt{rm.overwrite = True}

The \texttt{merge()} method accepts any list of Jobs, or a single Job
object. This includes both a simple list and a job slice or selection.
If any of the Jobs in the list contain subjobs then the merge object
will first be set on the Job and then merge called.

The user can use the same merge instance to merge multiple lists of jobs.


\section{Running \ganga jobs on the LHC Computing Grid}
\subsection{Introduction}
The LHC Computing Grid (LCG) is a world-wide distributed computing environment
developed for supporting the LHC experiments in High Energy Physics. The LCG
jobs are submitted from a User Interface (LCG-UI) and managed by a
Workload Management System (WMS).

The \ganga LCG backend handler implements the interactions with the WMS for
running jobs on LCG. It supports the two versions of the WMS, the EDG version
(EDG-WMS) and the gLite version (gLite-WMS).

\subsection{Getting start}
Running the \ganga LCG backend handler requires a LCG-UI environment. \ganga
allows user to specify alternative LCG-UI environments by specifying the configuration parameters,
\texttt{EDG_SETUP} and \texttt{GLITE_SETUP} corresponding to the UI setup for
accessing the EDG-WMS and the gLite-WMS, respectively.

By default, the module of the LCG backend handler is automatically loaded
with only the support for the EDG-WMS. The module first checks the availability
and the validity of the Grid credential. Credential renewal will be
triggered if the module cannot find a valid one to access LCG.   

To enable the support for the gLite-WMS, one can either set the parameter
\texttt{GLITE_ENABLE} to \texttt{True} within the \texttt{LCG} section of the
\texttt{.gangarc} configuration file before starting \ganga or directly enable it
inside a running \ganga session by:
\begin{verbatim}
In [1]:config['LCG']['GLITE_ENABLE'] = True
\end{verbatim}

As soon as the support for gLite-WMS is enabled, the module will check the
availability of the VOMS information attached with the Grid credential and
will force a credential re-initialization if no VOMS information is obtained.  

Running a simple LCG job through \ganga can be done as follows:
\begin{verbatim}
In [1]:j = Job(backend=LCG())

In [2]:j.submit()
Ganga.GPIDev.Lib.Job               : INFO     submitting job 117
Ganga.GPIDev.Adapters              : INFO     submitting job 117 to LCG backend
Ganga.Lib.LCG                      : INFO     using the VO defined as atlas.
Out[2]: 1

In [3]:jobs(117)
Out[3]: Job (
 status = 'completed' ,
 name = '' ,
 inputdir = '/afs/cern.ch/user/h/hclee/gangadir/workspace/Local/117/input/' ,
 outputdir = '/afs/cern.ch/user/h/hclee/gangadir/workspace/Local/117/output/' ,
 id = 117 ,
 outputsandbox = [] ,
 splitter = None ,
 merger = None ,
 inputsandbox = [ ] ,
 application = Executable (
    exe = 'echo' ,
    env = {} ,
    args = ['Hello World']
    ) ,
 outputdata = None ,
 subjobs = [ ] ,
 inputdata = None ,
 backend = LCG (
    status = 'Done (Success)' ,
    reason = 'Job terminated successfully' ,
    middleware = 'EDG' ,
    actualCE = 'gridce.roma3.infn.it:2119/jobmanager-lcgpbs-atlas' ,
    id = 'https://gdrb01.cern.ch:9000/JLiZPFLOD9zaa5aLNCITEQ' ,
    CE = None ,
    exitcode = '0' ,
    requirements = LCGRequirements (
       other = [] ,
       memory = None ,
       software = [] ,
       ipconnectivity = None ,
       cputime = None ,
       walltime = None
       )
    )
 )
\end{verbatim}

In the example, \ganga submits a simple "Hello World" job to LCG through the
EDG-WMS. During the job runtime, the \ganga monitoring loop will update the
backend attributes, \texttt{backend.status}, \texttt{backend.reason}, \texttt{backend.actualCE} and
\texttt{backend.id}, based on the job information retrieved from the LCG
Logging-and-Bookkeeping System (LCG-LB).

Once the job is reported as "completed", one can find the output in the directory of \texttt{outputdir}.
\begin{verbatim}
In [5]:ls $jobs(117).outputdir
__jobscript__.log  stderr  stdout

In [6]:cat $jobs(117).outputdir/stdout
Hello World
\end{verbatim}

One could switch \ganga to submit jobs through the gLite-WMS by simply
changing the attribute, \texttt{backend.middleware}. For example,
\begin{verbatim}
In [7]:j=Job(backend=LCG())

In [8]:j.backend.middleware = 'GLITE'

In [9]:j.submit()
Ganga.GPIDev.Lib.Job               : INFO     submitting job 118
Ganga.GPIDev.Adapters              : INFO     submitting job 118 to LCG backend
Ganga.Lib.LCG                      : INFO     using the VO defined as atlas.
Out[9]: 1

In [10]:j.backend
Out[10]: LCG (
 status = None ,
 reason = None ,
 middleware = 'GLITE' ,
 actualCE = None ,
 id = 'https://rb101.cern.ch:9000/nnp5ftGXe7DjADPLFGWbzg' ,
 CE = None ,
 exitcode = None ,
 requirements = LCGRequirements (
    other = [] ,
    memory = None ,
    software = [] ,
    ipconnectivity = None ,
    cputime = None ,
    walltime = None
    )
 )
\end{verbatim}

\subsection{\ganga configuration for the LCG backend handler}
The \ganga configuration parameters for the LCG backend handler are summarized
in the following tables.

The parameters related to the setup of the WMS and the UI:
\begin{tableiv}{l|l|l|l}{constant}{Parameter}{Description}{Data type}{Default value}
\lineiv{EDG_ENABLE}{enables the supports for the EDG-WMS}{boolean}{\texttt{True}}
\lineiv{GLITE_ENABLE}{enables the supports for the gLite-WMS}{boolean}{\texttt{False}}
\lineiv{EDG_SETUP}{sets the LCG-UI configuration for the EDG-WMS}{str}{the setup
script for CERN LCG-UI}
\lineiv{GLITE_SETUP}{sets the LCG-UI configuration for the
gLite-WMS}{str}{the setup script for CERN LCG-UI}
\lineiv{Config}{sets alternative WMS configuration for job submission}{str}{None}
\lineiv{ConfigVO}{sets alternative VO-specific WMS configuration for job
submission}{str}{None}
\end{tableiv}

The parameters related to the Grid credential properties:
\begin{tableiv}{l|l|l|l}{constant}{Parameter}{Description}{Data type}{Default value}
\lineiv{VirtualOrganisation}{specifies the name of the Virtual
Organisation}{str}{\texttt{dteam}}
%%\lineiv{ProxyTimeValid}{sets the validity of the Grid credential in minutes}{int}{\texttt{24*60}}
%%\lineiv{ProxyTimeLife}{sets the Grid credential to be invalid if its lifetime is
%%shorter than the given minutes}{int}{\texttt{1*60}}
\end{tableiv}

\begin{seealso}
    \seetext{More configurable Grid credential properties could be found in section 10, \textit{Credential management}.}
\end{seealso}

The parameters related to the LCG job description:
\begin{tableiv}{l|l|l|l}{constant}{Parameter}{Function}{Data type}{Default value}
\lineiv{AllowedCEs}{restricts the jobs to be submitted to the given list of
CEs}{JDL regular expression}{None}
\lineiv{ExcludeCEs}{restricts the jobs to be NOT submitted to the given list of
CEs}{JDL regular expression}{None}
\lineiv{RetryCount}{sets the retry number for the WMS scheduling}{int}{WMS
default}
\lineiv{MyProxyServer}{sets the hostname of the MyProxy server}{str}{UI
default}
\lineiv{Rank}{sets the expression for resource ranking}{JDL
expression}{None}
\end{tableiv}

\begin{notice}
\ganga will raise an \texttt{Exception} if the change on Virtual
Organisation is attampted in a single session.
\end{notice}

\subsection{Attributes of the LCG backend handler object}
The attributes of the \ganga LCG backend handler are summarized
in the following table.

\begin{tableiv}{l|l|l|l}{constant}{Attribute}{Function}{Data type}{Default value}
\lineiv{middleware}{sets the middleware type of the target WMS}{str}{EDG}
\lineiv{CE}{restricts the job to be run on the given CE}{str}{None}
\lineiv{requirements}{specifies the resource requirement}{LCGRequirement}{LCGRequirement default}
\end{tableiv}

The attributes of the \texttt{LCGRequirement} object:
\begin{tableiv}{l|l|l|l}{constant}{Attribute}{Function}{Data type}{Default value}
\lineiv{memory}{sets the minimum requirement of memory}{int}{None}
\lineiv{ipconnectivity}{sets if outbound connectivity is
required}{boolean}{None}
\lineiv{cputime}{sets the minimum requirement of cputime}{int}{None}
\lineiv{walltime}{sets the minimum requirement of walltime}{int}{None}
\lineiv{software}{sets the required software installation}{list}{[]}
\lineiv{other}{sets other requirements}{list}{[]}
\end{tableiv}

\section{Writing scripts}
You can take advantage of \ganga to write scripts that will perform a large
set of actions for you. As an example there is a script \texttt{submitjobs.py}
below that will submit a set of jobs corresponding to all scripts in a
directory \texttt{mydir} below your home directory:
\begin{verbatim}
import os
dir=os.environ['HOME']+'/mydir'
file=os.listdir(dir)
for line in file:
  print line
  j = Job(Executable(exe=File(dir+'/'+line))
  j.submit

\end{verbatim}
Note that you cannot use the \ipython short hand notations script. To run
this script you can either do it from within \ganga like
\begin{verbatim}
In [16]: execfile('submitjobs.py')
\end{verbatim}
or indeed straight from the shell command line
\begin{verbatim}
[lxplus061]~% ganga submitjobs.py
\end{verbatim}
If you do more that just a single thing there is an advantage in running from
within \ganga. If you run multiple scripts at the shell prompt you will for
each new script get the penalty of initialising your job repository.

\begin{seealso}

\ganga contains a few scripts for basic job operations such as submit, kill etc. The scripts
are available in {\tt ganga/python/Ganga/scripts} directory in each \ganga release.

CVS \ganga scripts via the Web:
\seeurl{http://isscvs.cern.ch/cgi-bin/viewcvs-all.cgi/ganga/python/Ganga/scripts/?cvsroot=ganga}{}

Atlas defines the {\tt athena} script for simplified submission of Athena jobs in {\tt ganga/python/GangaAtlas/scripts}.

CVS \ganga Atlas scripts via the Web:
\seeurl{http://isscvs.cern.ch/cgi-bin/cvsweb.cgi/ganga/python/GangaAtlas/scripts/?cvsroot=ganga}{}
\end{seealso}

The configuration  option {\tt [Configuration]  SCRIPTS_PATH } defines
the  search path  for the  scripts. The  current working  directory is
always searched first.

\section{Configuration}
\subsection{Command line options}
\ganga takes several command line arguments which are listed below.
\begin{verbatim}
[lxplus059]~% ganga --help
usage: 
*** Welcome to Ganga ***
Version: Ganga-4-2-1
Documentation and support: http:/cern.ch/ganga
Type help() or help('index') for online help.

usage: ganga [options] [script] [args] ...

options:
  --version             show program's version number and exit
  -h, --help            show this help message and exit
  -i                    enter interactive mode after running script
  --gui                 Run Ganga in the GUI mode.
  --config=FILE         read user configuration from FILE (default ~/.gangarc)
  --config-path=CONFIG_PATH
                        site/experiment config customization path, overrides the
                        GANGA_CONFIG_PATH variable
  -g, --generate-config
                        generate a default config file, backup the existing one
  -oEXPR, --option=EXPR
                        set configuration options, may be repeated mutiple
                        times,for example: -o[Logging]Ganga.Lib=DEBUG
                        -oGangaLHCb=INFO -o[Configuration]TextShell = IPython
                        FIXME: PATH-like variables are reset and not appended to
                        (unlike the config file behaviour)
  --quiet               only ERROR messages are printed
  --very-quiet          only CRITICAL messages are printed
  --debug               all messages including DEBUG are printed
  --no-prompt           never prompt interactively for anything except IPython
                        (FIXME: this is not fully implemented)
  --no-rexec            rely on existing environment and do not re-exec ganga
                        processto setup runtime plugin modules (affects
                        LD_LIBRARY_PATH)
\end{verbatim}

\subsection{Configuration file}

The configuration file will be covered here while the options related to
verbosity are covered in the next section. The remaining options are for
advanced use and will not be covered.

By default when \ganga starts  it tries to read the configuration file
\texttt{\~{}/.gangarc} but  as seen from  above a command  line argument
can change  this. The configuration file  itself is in  the INI format
and you can create an empty  one by running \texttt{ganga -g}. A small
section of a default configuration file is shown below.

\begin{verbatim}
# Ganga configuration file ($Name: Ganga-5-8-10 $). DO NOT remove this line.
#
:
:
#=======================================================================
[Configuration]

# The path to runtime plugin packages where you can add custom handlers
# Normally you should not worry about it.
# If an element of the path is just a name (like in the example below)
# then the plugins will be loaded using current python path. This means that
# some packages such as GangaAtlas or GangaLHCb may be taken from the release area
# Examples:
#RUNTIME_PATH = GangaAtlas
#RUNTIME_PATH = GangaLHCb:GangaAtlas
:
:
#=======================================================================
[Logging]

# Logging controls the Ganga output messages.
# The settings are applied hierarchically to the loggers.
# Ganga is the name of the top-level logger.
# The log level may be one of: CRITICAL ERROR WARNING INFO DEBUG
# Examples:
#Ganga = CRITICAL
#Ganga.Runtime = DEBUG
\end{verbatim}

The most  important parameter here is  the \texttt{RUNTIME_PATH}. This
decides which  directories are searched  for handlers. The  default is
decided by your group setup. Note, that in the automatically generated
user config  file all the settings  are commented out  (void) and thus
your group settings will be used. 

The environment  variable GANGA_CONFIG_PATH specifies  the sequence of
configuration   files  which   are  loaded   {\it  before}   the  user
configuration  file.  These  files contain  the your  group  and local
sitre settings.   The experiment-related settings are  defined in {\tt
python/GangaLHCb/LHCb.ini} and {\tt python/GangaAtlas/Atlas.ini}.

\subsection{Configuration in GPI}

You may also see the configuration options interactively using the built-in \texttt{config} object:

\begin{verbatim}
In [14]:print config
Ganga Configuration
Configuration                   : global configuration parameters.
DIRAC                           : Parameters for DIRAC
Display                         : control the printing style of the job registry ("print jobs")
File_Associations               : default associations between file types and file-viewing commands.
GPIComponentFilters             : Customization of GPI component object assignment
LCG                             : LCG/gLite/EGEE configuration parameters
LHCb                            : Parameters for LHCb
LSF                             : internal LSF command line interface
Local                           : parameters of the local backend (jobs in the background on localhost)
LocalAMGA_Repository            : Settings for the local AMGA job repository
LocalXML_Repository             : Parameters of the local XML-based job repository
Logging                         : control the messages printed by Ganga
...

In [15]:print config.Logging
Logging : control the messages printed by Ganga
The settings are applied hierarchically to the loggers. Ganga is the name of the top-level logger which
applies by default to all Ganga.* packages unless overriden in sub-packages.
You may define new loggers in this section.
The log level may be one of: CRITICAL ERROR WARNING INFO DEBUG

     Ganga = 'WARNING'
          top-level logger
          Type: <type 'str'>
     Ganga.GPIDev = 'INFO'
          logger of Ganga.GPIDev.* packages
          Type: <type 'str'>
     Ganga.Runtime.bootstrap = 'INFO'
          FIXME
          Type: <type 'str'>
     Ganga.Utility.logging = 'WARNING'
          logger of the Ganga logging package itself (use with care!)
          Type: <type 'str'>
     _colour = True
          enable ASCII colour formatting of messages e.g. errors in red
          Type: <type 'bool'>
     _format = 'NORMAL'
          format of logging messages: TERSE,NORMAL,VERBOSE,DEBUG
          Type: <type 'str'>
     _interactive_cache = True
          if True then the cache used for interactive sessions, False disables
          caching
          Type: <type 'bool'>
     _logfile = '~/.ganga.log'
          location of the logfile
          Type: <type 'str'>
     _logfile_size = 100000
          the size of the logfile (in bytes), the rotating log will never exceed
          this file size
          Type: <type 'int'>
\end{verbatim}

You may also change some configuration options but BEWARE: not all
configuration options may take effect when changed inside the Ganga
session. Option set at \ganga startup (in the configuration file or
via -o command line option) are marked with one asterisk (*) while
options changed interactively are marked with two asterisks (**):

\begin{verbatim}
In [16]:config.Logging._colour=False
In [17]:print config.Logging
...
**   _colour = False
          enable ASCII colour formatting of messages e.g. errors in
          red
          Type: <type 'bool'>
...
\end{verbatim}

You may also use TAB expansion to see the options. Caveat: options
starting with an underscore will not be presented as a tab expansion
choice. Some options, especially in the Logging section may contain
dots. Due to the rules of python syntax such names may not be
tab-expanded and you must use special syntax to access them:

\begin{verbatim}
In [1]:config.Logging.Ganga.GPIDev
---------------------------------------------------------------------------
exceptions.AttributeError                            Traceback (most recent call last)

/afs/cern.ch/user/k/kuba/<console>

AttributeError: 'str' object has no attribute 'GPIDev'

In [2]:config.Logging['Ganga.GPIDev']
Out[2]: INFO
\end{verbatim}



\section{Credential management}

\subsection{Introduction}

Ganga automatically checks the user credentials. If credentials, such
as AFS token, expire \ganga may no longer save job output files or
update the job information stored in the AFS directory. In such a case
\ganga will automatically freeze job output retrieval and switch to
the read-only mode. The command \texttt{reactivate()} may be used to
bring this functionality back once all required credentials have been restored.

At startup, \ganga creates the objects \texttt{gridProxy}
and \texttt{afsToken}, using which you can interact with your Grid Proxy
and AFS token respectively.  These credential objects have properties
that you can set in your \texttt{/.gangarc} configuration file, or modify
at the \ganga prompt.

\subsection{gridProxy properties, and automatic proxy checking}
Properties for the gridProxy object can be set in the configuration file as
in the following example:

\begin{verbatim}
[defaults_GridProxy]
# Default validity (hh:mm) with which Grid proxy should be created
validityAtCreation = 50:00
# Default minimum validity (hh:mm) for creating new proxy
minValididty = 20:00
# Maximum number of attempts allowed when entering passphrase
maxTry = 3
# Options to be passed to underlying command for creating proxy
init_opts =
# Virtual Organisation Management System to be used
voms = atlas
\end{verbatim}

In general, you should set the value of the \texttt{voms} property to
the name of your Virtual Organisation (VO), for example \texttt{atlas} or
\texttt{lhcb}.  In this case, \ganga will make use of VOMS-based commands
(for example, \texttt{voms-proxy-init} for proxy creation), and you will
have the permissions that have been set for you by your VO.  If you
specifically want \ganga to use non-VOMS commands (for example,
\texttt{grid-proxy-init} for proxy creation), then you should set
\texttt{voms} to a null value.

As part of its monitoring procedure, \ganga periodically checks
your Grid proxy's validity.  If this is less than the value
specified for \texttt{minValidity} then \ganga will prompt
you for creation of a new proxy.

\subsection{afsToken properties}

Properties for the afsToken object can be set in the configuration file as
in the following example:

\begin{verbatim}
[defaults_AfsToken]
# Default validity with which AFS token should be crreated
validityAtCreation = 12:00
# Default minimum validity (hh:mm) for creating new AFS token
minValididty = 00:15
# Maximum number of attempts allowed when entering password
maxTry = 2
# AFS username - if no value is set then login name is used
username = myName
# AFS cell - if no value is set then local AFS cell is used
# (cern.ch for users on lxplus.cern.ch)
cell = cern.ch
\end{verbatim}

\subsection{Actions with gridProxy and afsToken}

The following methods may be invoked for gridProxy and afsToken:

\begin{description}
\item[create] Create a new credential (Grid proxy or AFS totken).
\item[renew] Create a new credential, but only if the existing credential
is no longer valid.
\item[timeleft] Return remaining time for which credential is valid.
\item[isValid] Determine whether credential is valid.
\item[location] Return credential location (Grid proxy only).
\item[destroy] Destroy credential - to be used with care.
\end{description}

Details of the arguments that can be passed with each method call can be
found in \ganga's online help.  Some example usage is shown below.

\begin{verbatim}
# Examine gridProxy properties
In [1]:gridProxy 
Out[1]: GridProxy (
 validityAtCreation = '24:00' ,
 minValidity = '00:30' ,
 maxTry = 1 ,
 init_opts = '' ,
 voms = 'lhcb'
 )

# Determine time for which proxy remains valid
In [2]:gridProxy.timeleft()
Out[2]: 1:50:44 

# Create new proxy with default validity
# (i.e. as specified by validityAtCreation property)
In [3]:gridProxy.create()
Your identity: /C=UK/O=eScience/OU=Cambridge/L=UCS/CN=karl harrison
Enter GRID pass phrase:
Creating temporary proxy ......................................... Done
Contacting  lcg-voms.cern.ch:15003 [/C=CH/O=CERN/OU=GRID/CN=host/lcg-voms.cern.ch] "lhcb" Done
Creating temporary proxy .................................... Done
Contacting  lcg-voms.cern.ch:15003 [/C=CH/O=CERN/OU=GRID/CN=host/lcg-voms.cern.ch] "lhcb" Done
Creating proxy ..................................................... Done
Your proxy is valid until Wed Oct 18 11:58:22 2006
Ganga: INFO     Credential renewal process successful.
Out[3]: 1

# Determine time for which new proxy remains valid
In [4]:gridProxy.timeleft()
Out[4]: 23:59:24

# Create AFS token valid for 12 hours
In [5]:afsToken.create(validity="12:00")
Password:
Ganga: INFO     Credential renewal process successful.
Out[5]: 1

# Determine time for which AFS token is valid
# (initial actual validity can be larger than requested validity) 
In [6]:afsToken.timeleft()
Out[6]: 12:11:20
\end{verbatim}

\section{Verbosity of \ganga}
As is the case in many other software projects you can change the verbosity of
\ganga through the levels: CRITICAL, ERROR, WARNING, INFO, and DEBUG. This can
be done either at a global level or for specific parts of \ganga.

If you simply want on a case by case situation for \ganga to adjust the
verbosity you can use the command line options as listed in the extract below:
\begin{verbatim}
~% ganga -h
usage:
*** Welcome to Ganga-5 ***
Documentation and support: http:/cern.ch/ganga

usage: ganga [options] [script] [args] ...

options:
  -h, --help            show this help message and exit
  --quiet               only ERROR messages are printed
  --very-quiet          only CRITICAL messages are printed
  --debug               all messages including DEBUG are printed
\end{verbatim}

If you want to change the verbosity in a permanent way you can do this by
editing the \texttt{.gangarc} initialisation file. Look in the file itself for
documentation about how the logging can be turned up and down for individual
packages. 

For example, to turn DEBUG mode for GangaLHCb you may use any of the following methods:
\begin{enumerate}
\item change the configuration file \texttt{\~{}gangarc: [Logging]GangaLHCb=DEBUG} 
\item set command-line option: \texttt{ganga -o'[Logging]GangaLHCb=DEBUG'}
\item interactively: \texttt{[In 1]: config.Logging.GangaLHCb='DEBUG'}
\end{enumerate}


If you submit a bug to \ganga please run your session with full debugging
information for your documentation of the bug.

\section{Reporting bugs}
As with all other software reporting of bugs is an essential task any user can
help with. To report a bug to \ganga go to the page in Savannah.
\begin{seealso}
\seeurl{https://savannah.cern.ch/bugs/?group_id=195}{}
\end{seealso}
Check if the bug is already reported, and if not submit a new one. Please
include maximal information about how to reproduce the bug and if you include
output from the command line turn debug information on first (just start
\ganga with the \texttt{--debug} option).

\section{Extending \ganga}
\label{sec:Extending}
It is easy to extend \ganga to include a new plugin with \texttt{backends},
\texttt{applications} or \texttt{splitters} that are not part of the main
distribution. The extension can be done without making any changes in the core
part of the distribution but simply appending the \texttt{RUNTIME_PATH} in the
configuration file to where the new code is located.

Before you start on writing your own extensions think carefully about if the
same functionality can be achieved by creating a template job that simply
configures one of the existing plugins for your specific jobs.

\begin{seealso}
  The \emph{Developer Survival Guide} will give you many hints and
  instructions if you want to develop something yourself.
  \seeurl{http://ganga.web.cern.ch/ganga/work/Ganga4/survival/tut/tut.html}{}
\end{seealso}

The \lhcb and ATLAS extensions are implemented as plugins to the \ganga core
so you can use them as examples for how to develop your own plugin. Please get
in contact with the development team if you need help.

\section{Release Notes}
\label{ReleaseNotes}

\subsection{New features in \release}

\emph{This release is not backwards compatible}. The changes cover
the persistently saved data (job repository), configuration options and the GPI
interface.  Consult migration instructions in section \ref{sec:migration}.

The main purpose of this release is to clean-up the interface, to
introduce a number of long-awaited features and to prepare Ganga for
potential new features such as remote workspace.

The main changes in Ganga 5.0 release:
\begin{itemize}
\item improved configuration interface and restructed some of the configuration options
\item support for job selections, slicing and collective operations
\item usability improvement: summary print and \texttt{full_print()}
\item optimized job repository for memory and speed
\item changed default location of the local workspace and repository (still under \~{}/gangadir)
\item type checking of job parameters and configuration options
\end{itemize}

\subsubsection{\Atlas}
The main changes for \Atlas experiment include:
\begin{itemize}
\item Athena
\begin{itemize}
\item DQ2Dataset and all other routine support multiple datasets
\texttt{j.inputdata.dataset=['datasetname1','datasetname2']}
\item \texttt{DQ2Dataset.type='DQ2_COPY'} Copies input files and starts athena on the worker node one-by-one
\item many small updates and bug-fixes
\end{itemize}

\item AthenaMC
\begin{itemize}
\item Added new production mode, template, to cover all transformations that
  cannot be run with the three existing modes.
\item Added new production mode, "template", to handle any transformation
  not covered by the three existing modes (evgen, simul, recon).
\item AthenaMCInputDatasets has been extended to 4 new members, 2 related to
  input minimum bias and 2 related to cavern noise, to allow the use of
  the pileup transformation.
\item many small updates and bug-fixes
\end{itemize}
\item Tasks
\begin{itemize}
\item many updates and sync with Ganga 4.4.x developments
\end{itemize}
\end{itemize}

\subsubsection{\lhcb}

The main changes for \lhcb experiment include:
\begin{itemize}
\item Options files can now be specified in Python format as well.
\item The optsfile attribute of the Gaudi applications in now a list.
If more than one element is given to the list they are simply
concatenated. Ex \texttt{j.application.optsfile=['file1.py','file2.py]}
Mixing of old text format and new Python format is not allowed.
\item The platform of an application can now be specified.
\texttt{j.application.platform='slc4_ia32_gcc34'}. This will override the
CMTCONFIG value. Make sure you have your shared libraries
compiled for the platform you submit with!
\end{itemize}

\subsection{New features in \ganga-4.2}
Compared to the last production release \ganga-4.1 the following new features are available:
\begin{itemize}
\item optimized bulk submission and handling of the input sandboxes
\item new GangaPlotter utility and support for Root application and Interactive/PBS/SGE backends
\item \lhcb bookkeeping database browser (enable with GangaLHCb and GangaGUI in the RUNTIME_PATH)
\item \lhcb optimization in Dirac submission (shared sandboxes)
\item Atlas DDM/DQ2 dataset integration and complete support for Athena analysis
\item Atlas support for private Monte-Carlo productions (AthenaMC)
\end{itemize}

\subsection{New featues in \ganga-4.1}
Compared to the last production release \ganga-4.0.2 the following new
features are available:
\begin{itemize}
\item A \texttt{jobtree} to support a logical directory structure for jobs.
\item Graphical User Interface available. Start it with
\begin{verbatim}
     ganga --gui
\end{verbatim}
  See separate manual for usage.
\item Implementation of sub jobs to analyse large datasets as a single logical
  job.
\item Ten times faster submission of \lhcb analysis if implemented through sub
  jobs.
\item Many improvements and bug fixes in interaction with Dirac backend.
\item Migration to production version of remote repository.
\item PBS and Condor backend handlers added.
\item Support for default properties to be set in configuration file.
\item Possibility to check out and compile \lhcb code from within \ganga
  session.
\item Direct submission of \lhcb jobs to LCG.
\item Rewritten local backend handler.
\end{itemize}

Some anticipated features did not make it into this release:
\begin{itemize}
\item Support for automatic merging of output from sub jobs
\item Improved treatment of grid proxy and afs token handling.
\item Direct access to the bookkeeping databases of the experiments.
\end{itemize}

\end{document}
